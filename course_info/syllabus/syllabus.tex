\documentclass[10pt]{article}

\usepackage{titlesec}
\titlespacing\section{0pt}{5pt}{2pt}
\titlespacing\subsection{0pt}{5pt}{2pt}
%\setlength{\parindent}{0pt} % remove automatic indentation


\usepackage[text={6.5in,8.5in},centering]{geometry}
\geometry{verbose,a4paper,tmargin=2.4cm,bmargin=2.4cm,lmargin=2.4cm,rmargin=2.4cm}

\usepackage{soul} % for \st{} strike-out
\usepackage{hyperref} % for activation hyperlinks
\hypersetup{colorlinks = true, allcolors = blue  }

\usepackage{enumitem}



%%%%%%%%%%%%%%%%%%%%%%%%%%%%%%%

\title{Syllabus:\\IB 516 ANALYTICAL WORKFLOWS}

\author{}
\date{}

%%%%%%%%%%%%%%%%%%%%%%%%%%%%%%%

\begin{document}
\maketitle
\vspace{-50pt}


\section*{Course Description}
	Have you proposed a modeling chapter for your dissertation but need support getting things up and
	running?
	Are you sitting on a data set ready for analysis and visualization but don't know how or
	where to begin?
	Maybe you're far along in some series of analyses and feel ``lost in the trees.''

	This course will help you with these challenges by practicing the development and implementation of
	efficient, reproducible \emph{workflows} for your projects.
	Every project should (and can) be modular and fully automated, hence reproducible, portable and
	easily modified.
	Rerunning a model under a different set of parameters should (and can) be as simple as a few
	keystrokes.
	Regenerating all analyses, figures and tables after finding a typo in your code or dataset should (and
	can) be painless.

	Efficient workflows start at project conception and end only if the project idea is itself a dead end.
	Thus, in this course, we'll work to practice
	(1) refining and articulating project ideas and goals,
	(2) creating modular and automated analyses, and
	(3) using best-practices in coding and project management.
	We'll learn how to use Git, GitHub, \LaTeX\, and Markdown.
	The instructors will mostly use \textsf{R} within RStudio, but users of other programming languages
	and text editors are welcome and encouraged.
	You will need either
	(1) a large unwieldy dataset and an end goal (e.g., reproducing someone else's analysis or
	visualization) or
	(2) a dynamical model or simulation (or sufficiently well-developed ideas for one).
	The use of other people's data or published models is also encouraged, as needed.

\section*{Prerequisites}
\noindent
	Officially, only graduate standing (or instructor approval).
	But see end of \emph{Course Description} for what you should have coming in.

\section*{Learning Outcomes}
\noindent
After successful completion of this course, you should be able to:
\begin{enumerate}
	\itemsep0em
	\item Translate a research plan into an explicit analytical workflow;
	\item Apply best practices in scientific programming to construct reproducible research;
	\item Manage and collaborate on complex research projects using a version control system;
	\item Apply analytical workflows to advance your dissertation research.
\end{enumerate}

\section*{Meeting Times and Location}
\noindent
	Tuesdays and Thursdays, 10-11:50am\\
	Fall 2022: Kidder 356

\section*{Instructors}
\noindent
	Ben Dalziel\\
	Email: \href{mailto:benjamin.dalziel@oregonstate.edu}{benjamin.dalziel@oregonstate.edu}\\
	Phone: 541 737 1979\\
	Office: CRB 2128\\

	\noindent
	Mark Novak\\
	Email: \href{mailto:mark.novak@oregonstate.edu}{mark.novak@oregonstate.edu}\\
	Phone: 541-737-3610\\

\section*{Office Hours}
\noindent
	% I am happy to meet by appointment, or please feel free to call or stop by my office.
	We are happy to meet with you at any point in the quarter. Please use email to set up an appointment, or we can setup a time in class.

	\clearpage
\section*{Course Materials \& Schedule}
\noindent
\begin{center}
\textbf{See
\href{https://github.com/analyticalworkflows}{https://github.com/analyticalworkflows/}
 for all materials and the schedule of daily readings and assignments.}
\end{center}

\section*{Course Work}
This course combines almost 4 hours of weekly instruction and activities for a total of 4.0 credits.
There will be no exams, tests, or quizzes.
We will not grade your work because there won't be ``correct'' answers to the problems you will
solve.
In fact, my goal is for you to not have almost no ``homework'' (besides readings and making progress
on your project).
However, the course \textit{will} require a significant allocation of time for thought and reflection outside
of class.
There \textit{will} also be readings for in-class discussion to be read before class (see Schedule on GitHub repository).
Finally, you will need to complete all assigned in-class work (e.g., challenges and Git commits), ideally
by the end of the week in which the topic is covered but definitely before the end of Week 10.

\section*{Evaluation of Student Performance}
Grading (A-F) will be based on your participation, your project presentations (see rubrics on the Github repository),
and the completion of the various in-class ``tasks'' assigned throughout the quarter.
Accommodations for missed classes (e.g., due to fieldwork) are okay in principle, as long as we can identify a path to your having fully participated in the course.
Please discuss your specific situation with me during Week 1.
Everyone who puts the effort into the class will get a grade that will make them happy (if you, as a graduate student, still care about grades).

\section*{Collaboration}
My hope is for you to make a tremendous amount of progress on your chosen (and ideally, thesis-related) project during this course.
Your work will therefore, almost by definition and necessity, be your own.
That said, we will \textit{strongly} encourage everyone to work collaboratively in whatever way possible!
That's because what this course is all about is to provide you with tools.
Tools have to be learned and practiced.
They can't be plagiarized, and you won't get anything out of them (or this course) if you try.
Moreover, there's nothing better for learning to understand something than (successfully) explaining how it works to someone else.
Therefore, you are \textit{strongly} encouraged to collaborate
This will be emphasized throughout the quarter by getting you to work or discuss your work in pairs or teams.

%\section*{Schedule}
%\begin{tabular}{clll}
%	\hline\hline
%	Wk  & Day & Date & Topic \\
%	\hline
%	\hline
%	0 &  Th		& 9/24 		& Course organization \& Philosophy of workflows \\
%	1 & T  		& 9/29 		&  Project setup \& Version control (Git) \\
%	& Th 	& 10/1 		& Project presentations	\\
%	2 &  T 		& 10/6		& b		\\
%	&  Th 	& 10/8 		&b   \\
%	3 &  T		 & 10/13 	& b  \\
%	&  Th 	& 10/15		& b\\
%	4 &  T 		& 10/20		& b  \\
%	&  Th 	& 10/22		& b \\
%	5 &  T 		& 10/27		& b\\
%	&  Th 	& 10/29		& b \\
%	6 &  T 		& 11/3		& \textit{Election day!}  \\
%	&  Th	& 11/5		 & b\\
%	7 &  T 		& 11/10		& b \\
%	&  Th 	& 11/12		& b\\
%	8 &  T 		& 11/17		& b\\
%	&  Th 	& 11/19		& b  \\
%	9 &  T 	& 11/24		& b \\
%	&  Th 	& 11/26		& \multicolumn{1}{c}{ \textit{Thanksgiving - no class} } \\
%	10 &  T 	& 12/1		& Project presentations \\
%	&  Th 	& 12/3		&  Project presentations \& Wrap-up \\
%	\hline
%	\hline
%\end{tabular}

%\pagebreak

\clearpage
\section*{Course Policies}

\subsection*{Academic Calendar}
All students are subject to the registration and refund deadlines as stated in the Academic Calendar: 
\url{https://registrar.oregonstate.edu/osu-academic-calendar}

\subsection*{Equity, Justice and Inclusion}
Oregon State University, and the Department of Integrative Biology, have a lot of work to do on
improving the level of equity, justice and inclusion in our community.
We, the instructors and students in this class, will take responsibility for creating an welcoming and supportive climate for everyone in the course, and especially for those who are underrepresented and marginalized in academia and in our society. If you have concerns or suggestions related to these issues as they pertain to the course, please do not
hesitate to contact me and/or the Office for Institutional Diversity
(\href{https://diversity.oregonstate.edu/}{https://diversity.oregonstate.edu/}).

\subsection*{Student Conduct Expectations}
Choosing to join the Oregon State University community obligates each member to a code of responsible
behavior, which is outlined in the Student Conduct Code
(\href{https://beav.es/codeofconduct/}{ https://beav.es/codeofconduct/}).
This Code is based on the assumption that all persons must treat one another with dignity and respect
in order for scholarship to thrive.

\subsection*{Students with Disabilities}
Accommodations for students with disabilities are determined and approved by Disability Access
Services (DAS).
If you, as a student, believe you are eligible for accommodations but have not obtained approval, please
contact DAS at 541-737-4098 or at
\href{http://ds.oregonstate.edu}{http://ds.oregonstate.edu}.
DAS notifies students and faculty members of approved academic accommodations and coordinates
implementation of those accommodations.
While not required, students and faculty members are encouraged to discuss details of the
implementation of individual accommodations.

\subsection*{Reach Out for Success}
University students encounter setbacks from time to time. If you encounter difficulties and need assistance, it’s important to reach out. Consider discussing the situation with an instructor or academic advisor. Learn about resources that assist with wellness and academic success at oregonstate.edu/ReachOut. If you are in immediate crisis, please contact the Crisis Text Line by texting OREGON to 741-741 or call the National Suicide Prevention Lifeline at 1-800-273-TALK (8255)

\subsection*{Religious Holidays}
Oregon State University strives to respect all religious practices.
If you have religious holidays that are in conflict with any of the requirements of this class, please see me as soon as you can so that we can make alternative arrangements.

\subsection*{Student Learning Experience Survey}
During Fall, Winter, and Spring term the online Student Learning Experience surveys open to students the Wednesday of week 9 and close the Sunday before Finals Week. Students will receive notification, instructions, and the link through their ONID email. They may also log into the survey via MyOregonState or directly at \url{https://beav.es/Student-Learning-Survey}. Survey results are extremely important and are used to help improve courses and the learning experience of future students. Responses are anonymous (unless a student chooses to ``sign'' their comments, agreeing to relinquish anonymity of written comments) and are not available to instructors until after grades have been posted. The results of scaled questions and signed comments go to both the instructor and their unit head/supervisor. Anonymous (unsigned) comments go to the instructor only.



\end{document}
