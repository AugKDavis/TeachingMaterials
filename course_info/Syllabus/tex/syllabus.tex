\documentclass[]{article}

\title{Syllabus:\\
	IB 599 ANALYTICAL WORKFLOWS \\
	Developing Analytical Workflows for Ecological Models and Data Visualization}
\author{}

\begin{document}
\maketitle



\section*{Course Description}
	Have you proposed a modeling chapter for your dissertation but need support getting things up and running?  Are you sitting on a giant data set ready for visualization but don't know how to begin?  Maybe you're far along in some complex analysis and feel “lost in the trees.”
	
	This class will help you with these challenges by practicing efficient, reproducible workflows for your projects.  For example, every large project should be modular and fully automated, hence reproducible, portable and easily modified.  Rerunning a model under a different set of parameters should be as simple as a few keystrokes. Regenerating all analyses, figures and tables after finding a typo in your giant dataset should be painless.
	
	Efficient workflows start at project conception and end only if the project idea is itself a dead end.  Thus, in this class, we'll work to practice (1) refining and articulating project ideas and goals, (2) creating modular and automated analyses, and (3) using best-practices in project management. We'll learn the basics of Git (and possibly Markdown and/or LaTeX).  The instructors will mostly use R (RStudio), but users of other languages are
welcome.  You'll need either (1) a large unwieldy dataset and an end goal (e.g., reproducing someone's analysis or visualization) or (2) a dynamical model (or developed ideas for one). The use of other people's data or published models, if needed, is encouraged.

\section*{Instructor}
	Dr. Benjamin Dalziel\\
	Email: benjamin.dalziel@oregonstate.edu\\
	Phone: 541 737 1979\\
	Office: Cordley 5006\\

\section*{Office hours}
	I am happy to meet by appointment, or please feel free to call or stop by my office. 

\section*{Prerequisites}
	Officially, none. But see Course Description for what you should have going in: “(1) a large unwieldy dataset and an end goal (e.g., reproducing someone's analysis or visualization) or (2) a dynamical model (or developed ideas for one). The use of other people's data or
published models, if needed, is encouraged.”

\section*{Course work}
	This course combines 20 hours of instruction and activities for 2.0 credits. Crucially, the class will require significant allocation of time outside of class. We expect it will require about 2 hours of out of class work for each hour of seminar, or 4 hours of out of class per week.
	
\section*{Schedule}

\begin{tabular}{p{0.15\linewidth}p{0.45\linewidth}p{0.4\linewidth}} \hline
	Week (time) & Topic & To be completed before class\\ \hline
	
	1 
(1hr) & Organizational \& Overview \\
	
	2 
(2hrs) &Initial presentations (~10 min. each) & A 10 minute presentation on your proposed project.  See Initial Presentation Rubric on the GitHub page \\
	
	3 
(2hrs) & Version control \& coding best practices, Project setup \& organization (lecture) & Readings (posted by end of week 1), Setup Git and Rstudio on your computer, with any necessary troubleshooting completed beforehand \\
	
	4 
(2hrs) & ``Studio time'' (work on project) \\
	
	5 
(2hrs) & Workflow diagrams (1hr) \&  intro to \LaTeX  \\
	
	6 
(2hrs) & ``Studio time'' (work on project) \\
	
	7 
(2hrs) & Visualization workflows  & Readings, sketched visualization(s) for your project \\
	
	8 
(2hrs) & ``Studio time'' (work on project) \\
	
	9 
(2hrs) & ``Studio time'' (work on project) \\
	
	10 
(3hrs) & Final presentations & Created a presentation of your pipeline, key results,  visualizations
\end{tabular}

\section*{Course materials}
All available on https://github.com/analyticalworkflows

\section*{Evaluation of Student Performance}
Grading (A-F) will be based on: initial and final presentations (see Rubric on the Github repo), and participation. Accommodations for missed classes (e.g. due to fieldwork) are OK in principle, as long as we can identify a path to fully participating in the course. Please discuss your specific situation with the instructor during Week 1. I expect that everyone who puts effort into the class will get a grade they are happy with.


\section*{Course Policies}

\subsection*{Equity Justice and Inclusion}
Oregon State University, and the Department of Integrative Biology, have work to do on improving the level of equity, justice and inclusion in our community. The instructors and students in this class will take responsibility for creating an affirming climate for everyone in the course, and especially for those who are underrepresented and marginalized in academia and in our society, because of their race, creed, color, sex, religion, national origin, citizenship status, ancestry, genetic information, pregnancy, marital status, domestic partnership status, familial status, age, body size, education level, disability, veteran status, sexual orientation, and gender identity or expression. If you have concerns or suggestions related to these issues as they pertain to the course, please do not hesitate to contact me and/or the Office for Institutional Diversity (https://diversity.oregonstate.edu/).

\subsection*{Student Conduct}
Choosing to join the Oregon State University community obligates each member to a code of responsible behavior, which is outlined in the Student Conduct Code, available at http://oregonstate.edu/studentconduct/offenses-0.  This Code is based on the assumption that all persons must treat one another with dignity and respect in order for scholarship to thrive.  

\subsection*{Statement Regarding Students with Disabilities}
Accommodations for students with disabilities are determined and approved by Disability Access Services (DAS). If you, as a student, believe you are eligible for accommodations but have not obtained approval please contact DAS immediately at 541-737-4098 or at http://ds.oregonstate.edu. DAS notifies students and faculty members of approved academic accommodations and coordinates implementation of those accommodations. While not required, students and faculty members are encouraged to discuss details of the implementation of individual accommodations.

\subsection*{Religious Holidays}
Oregon State University strives to respect all religious practices.  If you have religious holidays that are in conflict with any of the requirements of this class, please see me immediately so that we can make alternative arrangements.


\end{document}
