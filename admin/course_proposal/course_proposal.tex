\documentclass[10pt]{article}
\usepackage[latin1]{inputenc}
\usepackage{amsmath,amsfonts,amssymb}
\usepackage{enumitem}
\usepackage{hyperref}
\hypersetup{
	colorlinks=true,
	linkcolor=blue,
	filecolor=magenta,      
	urlcolor=blue,
}
\usepackage{soul} % for strike-out

\usepackage{titlesec}
\titlespacing\section{0pt}{12pt plus 4pt minus 2pt}{0pt plus 2pt minus 2pt}

\usepackage[text={6.5in,8.5in},centering]{geometry}
\geometry{verbose,a4paper,tmargin=2.4cm,bmargin=2.4cm,lmargin=2.4cm,rmargin=2.4cm}

\author{Mark Novak \& Ben Dalziel}
\title{\emph{Course Proposal:}\\Analytical Workflows\\\emph{IB 5XX}}
\date{}

\begin{document}

\maketitle

\section*{Course Motivation}
Graduate students are amassing ever larger datasets to ask ever more challenging questions with ever 
more sophisticated  statistical techniques. 
Many others propose dissertation chapters entailing computationally demanding simulations or the 
analysis of parameter-rich mathematical models. 
Keeping pace in their field often requires students to straddle empirical, statistical and theoretical 
techniques. 
These multi-faceted and ever more collaborative approaches lie at the heart of our 
department's \emph{integrative} approach to biology and have become standard practice in many of the 
career paths our students pursue.
And yet, nowhere in our curriculum are students provided the opportunity to learn the skills, basic tools 
and best practices with which to manage the complexity of their projects.\footnote{See \emph{Related 
Courses at OSU} below.}
This poses a non-trivial challenge to our students that is fundamentally distinct from the learning of 
specific content, methods or platforms.

Our course provides students the skills and tools they need to organize and manage their projects from 
inception to publication.  
Working with their own data or models, students learn and practice: (re)organizing projects into 
efficient, reproducible and easily-modified ``\emph{analytical workflows}''; 
writing modular, transparent computer code in the programming language of their choice;
and tracking and communicating progress and hurdles to others using stand-alone and cloud-based 
version control and collaboration software.
Although these have proven to be game-changing for graduate students at all levels, the 
target audience for this course is students who have substantial data in hand or have an 
implementable framework for their analyses established (i.e. $2^{nd}$ to $4^{th}$ year Ph.D. and 
$2^{nd}$ to $5^{th}$ term M.S. students).

\section*{Student-targeted Advertisement}
Have you proposed a modeling chapter for your dissertation but need support getting things up and 
running?  
Will you soon be sitting on a complete data set ready for your planned analyses but don't know how or 
where to begin?
Maybe you're far along in a series of analyses and feel ``lost in the trees.''

This course will help you with these challenges by practicing the development and implementation of 
efficient, reproducible \emph{workflows} for your projects.  
Every project should (and can) be modular and fully automated, hence reproducible, portable and easily 
modified.  
Rerunning an analysis with a different set of parameters should (and can) be as simple as a few 
keystrokes. 
Regenerating all figures and tables for your manuscript after finding a typo in your code or dataset 
should (and can) be painless.

Efficient workflows start at project conception and end only if the project idea is itself a dead end.  
Thus, in this course, we'll work to practice 
(1) refining and articulating project goals and benchmarks, 
(2) creating modular and automated analyses, and 
(3) using best practices in coding and project management. 
We'll learn how to use Git, GitHub, \LaTeX\,, Markdown, and High Performance Clusters.
The instructors will mostly use \textsf{R} within RStudio, but users of other programming languages and 
text editors are welcome and encouraged.  
You will need either 
(1) a dataset and a visualization or analysis goal, or 
(2) a model or simulation (or sufficiently well-developed ideas for one).
The use of other people's data or published models is also encouraged, as needed.

\section*{Course Catalog Description}
Theory and implementation of efficient, reproducible workflows -- including best practices in scientific 
programming, project management, and collaboration --  for computational, analytical, and data-driven 
biological research.

\section*{Student Testimonials}
We have taught two prior IB 599 versions of this course: Spring 2019 (2 \emph{cr.}) \& Fall 2020 (4 
\emph{cr.}).
Students ranged from $1^{st}$ year M.S. to $5^{th}$ year Ph.D. from across three colleges. 
Student eSET scores were 6.0 and 6.0 in 2019 and 5.8 and 5.8 in 2020 for ``the course as a whole'' and 
``the instructor's contribution'' respectively\footnote{By contrast, the department's median scores were 
4.9 and 5.1 in Spring 2019, and 5.1 and 5.4 in Fall 2020, out of 6.0 points possible.}.
The following are excerpts of the feedback that was received via email and the eSET evaluations.

\begin{quote}
	\emph{I am of the opinion that this Analytical Workflow class is indispensable. Engaging in research 
	comes with a massive deluge of papers, files, data, output, etc, and yet prior to this class I had never 
	before been exposed to organizational best practices recommendations.}
\end{quote}
\begin{quote}
	\emph{I started this course with nothing but raw data and anxiety, and I am leaving with code that 
		works, a reproducible workflow, and the confidence and resources to press on.}
\end{quote}
\begin{quote}
	\emph{I'm so happy this class exists, and it was instrumental to much of the progress I've made this 
	term.}
\end{quote}
\begin{quote}
\emph{Ben and Mark helped turn a daunting task that I'd been putting off [...] into a 
well-organized reality! Five stars.}
\end{quote}
\begin{quote}
	\emph{This class has been extremely useful for the conceptual organization and technical execution 
	of my research. Thank you for organizing and leading this class!}
\end{quote}
\begin{quote}
	\emph{Who should take this course?  Anyone working on a project that uses coding or modeling.}
\end{quote}
\begin{quote}
	\emph{I really appreciate that you shared your experience and learning process with us (metacognitive 
		reflection! best teaching practices!).}
\end{quote}
\begin{quote}
	\emph{After taking the class, [...] I feel empowered to solve issues with my model on my own.}
\end{quote}
\begin{quote}
	\emph{It was reassuring and encouraging to be able to share feelings and experiences with my 
	colleagues and have them validated, listen to their counsel, and know that I am not the only one 
	facing these challenges.}
\end{quote}


\section*{Details}
\begin{tabular}{rl}
	\emph{Credits:}	&  4\\
	\emph{Frequency:} & Annually\\
	\emph{Quarter:} & Winter\\
	\emph{Course times:} &  Tuesday \& Thursday 10:00-11:50am\\
	\emph{Instructors:} & Mark Novak \& Ben Dalziel (of record alternating annually)\\
	\emph{Prerequisites}: & Graduate standing, or by instructor permission\\
	\emph{Enrollment cap}: & 18\\
\end{tabular}

\section*{Learning Outcomes}
After successful completion of this course, students will be able to:
\begin{enumerate}
	\itemsep0em
	\item Translate a research plan into an explicit analytical workflow;
	\item Apply best practices in scientific programming to construct reproducible research;
	\item Manage and collaborate on complex research projects using a version control system;
	\item Apply analytical workflows to advance their dissertation research.
\end{enumerate}

\section*{Course Philosophy}
Our primary goal in this course is for students to develop more efficient research skills.  
An important secondary goal is to have students make significant progress on their thesis work.  
Our philosophy is that students can achieve both because our primary goal is best achieved by having 
students practice new tools while working on their own research.

\section*{Course Materials}
For complete access to all teaching materials and learning resources, see\\
\url{https://github.com/analyticalworkflows/TeachingMaterials}.

\section*{Schedule}
\begin{enumerate}[label=\bfseries Week \arabic*:,leftmargin=*,labelindent=1em]
	\itemsep0em
	\item Course overview \& Philosophy\\
				Structuring projects \& Version control with Git
	\item Project proposals\\
				Workflow diagrams
	\item Coding best practices \\
				Hack-a-thon
	\item Git w/ GitHub (\emph{Project management \& collaboration})\\
				Hack-a-thon
	\item Typesetting with Markdown\\
				Data visualization
	\item Project progress presentations\\
				Hack-a-thon
	\item Faster computing (\emph{Vectorization \& parallel computing})\\
				Hack-a-thon
	\item Faster computing (\emph{High performance computing})\\
				Hack-a-thon
	\item Typesetting with \LaTeX\\\
				Hack-a-thon
	\item Project presentations\\
				Project presentations \& Wrap-up
\end{enumerate}


\section*{Related courses at OSU}
OSU's relevant statistics courses focus exclusively on data visualization.
Existing omics and informatics courses introduce students to specific analysis 
pipelines in particular programming environments and do not address issues relating to project 
management as a whole.
Existing courses are also not relevant to the many IB students pursuing non-\emph{omic} research.

\vspace{1pc}
\noindent
\textbf{BB 485/585 - Applied Bioinformatics}\\
Fundamental concepts needed to understand the software and methods used in bioinformatics. Includes 
contemporary techniques such as databases, gene and genome annotations, functional annotations, 
sequence alignment, motif finding, secondary structure prediction, phylogenetic tree construction, 
high-throughput sequence data, ChIP-Seq peak identification, transcriptome profiling by RNA-Seq, 
microRNA discovery and target prediction.  \emph{Prerequisites}: BB 314 or BB 314H.

\vspace{1pc}
\noindent
\textbf{ST 537 - Data Visualization} \emph{E-campus only}\\
Perceptual principles for displaying data; critique and improvement of data visualizations; use of color in 
visualization; principles of tidy data; strategies for data exploration; select special topics. 
\emph{Prerequisites}: ST 512 or ST 517 or ST 552.

\end{document}
