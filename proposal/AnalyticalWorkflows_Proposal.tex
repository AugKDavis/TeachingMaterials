\documentclass[10pt]{article}
\usepackage[latin1]{inputenc}
\usepackage{amsmath}
\usepackage{amsfonts}
\usepackage{amssymb}
\usepackage{graphicx}
\usepackage[text={6.5in,8.5in},centering]{geometry}
\geometry{verbose,a4paper,tmargin=2.4cm,bmargin=2.4cm,lmargin=2.4cm,rmargin=2.4cm}

\author{Mark Novak \& Ben Dalziel}
\title{Course Proposal:\\Analytical Workflows (IB 5XX)}

\begin{document}

\maketitle
\date{}

\section*{Motivation}

\section*{Student-targeted advertisement}
Have you proposed a modeling chapter for your dissertation but need support getting things up and running?  Are you sitting on a giant data set ready for analysis and visualization but don't know how to begin?  Maybe you're far along in some complex analysis and feel ``lost in the trees.''

This class will help you with these challenges by practicing efficient, reproducible workflows for your projects.  For example, every project should be modular and fully automated, hence reproducible, portable and easily modified.  Rerunning a model under a different set of parameters should (and can) be as simple as a few keystrokes. Regenerating all analyses, figures and tables after finding a typo in your code or giant dataset should (and can) be painless.

Efficient workflows start at project conception and end only if the project idea is itself a dead end.  Thus, in this class, we'll work to practice (1) refining and articulating project ideas and goals, (2) creating modular and automated analyses, and (3) using best-practices in coding and project management. We'll learn the basics of Git, \LaTeX\ and Markdown.  The instructors will mostly use R within RStudio, but users of other languages are
welcome and encouraged.  You will need either (1) a large unwieldy dataset and an end goal (e.g., reproducing someone else's analysis or visualization) or (2) a dynamical model or simulation (or sufficiently well-developed ideas for one). The use of other people's data or published models is also encouraged, as needed.

\section*{Course description}
\emph{Credits:} 4\\
\emph{Quarter:} Winter\\
\emph{Course times:} Tuesday \& Thursday 10:00-11:50\\
\emph{Frequency:} Annually (starting 2021)\\
\emph{Instructor of record:} Mark Novak, Ben Dalziel (alternating)\\
\emph{Prerequisites}: Graduate standing, or by instructor permission\\


\section*{Student testimonials}
A preliminary IB599 version of this course was taught by Ben Dalziel (with support from Mark Novak) in the Spring of 2019.  The following are excerpts of feedback that was received via email and eSET evaluations.

\begin{quote}
\emph{``Ben and Mark helped turn a daunting task that I'd been putting off (my modeling chapter) into a well-organized reality! Five stars.''}
\end{quote}
\begin{quote}
	\emph{``I wasn't experienced in R/Git or savvy about best practices in coding or reproducible workflows. Ben and Mark helped me structure my independent work and provided essential information about programs and best practices in coding.''}
\end{quote}
\begin{quote}
	\emph{``After taking the class, [...] I feel empowered to solve issues with my model on my own.''}
\end{quote}
\begin{quote}
	\emph{``I really appreciate that you shared your experience and learning process with us (metacognitive reflection! best teaching practices!).''}
\end{quote}

\end{document}